\maketitle

\addtocontents{toc}{\hspace{-7.5mm} \textbf{Capítulos}}
\addtocontents{toc}{\hfill \textbf{Página} \par}
\addtocontents{toc}{\vspace{-2mm} \hspace{-7.5mm} \hrule \par}

\tableofcontents

\maketitle



\maketitle

\section{Introduccion}

El presente ensayo busca entender que es la filosofia de la ciencia y busca responder las siguientes preguntas:

\begin{enumerate}
    \item ¿Cuales son los problemas de la filosofia de la ciencia?.
    
    \item ¿Cuales son las delimitaciones del ambito de los estudios de la ciencia?.
    
    \item ¿Cuales son las delimitaciones del ambito de los estudios de la Tecnologia?.
    
    \item ¿Cuales son las delimitaciones del ambito de los estudios de la Sociedad?.
    
    \item ¿Como se relacionan estos ambitos de estudio con los problemas de la Filosofia de las Ciencia?.
    
\end{enumerate}

\section{Conceptos}

\subsection{Filosofia de la ciencia}

Es un campo de investigacion que consiste en el analisis y la critica sobre la ciencia, analizando el impacto que genera esta rama del conocimiento en la sociedad, el medio ambiente y en la tecnologia. De forma que ayude, mediante el razonamiento, a guiar a los individuos sobre hacia en donde encaminar los conocimientos adquiridos y que beneficios traeran estos y su practica. En pocas palabras consiste en el cuestionamiento de como hacer ciencia y el por que hacerla en base a las causas que puedan traer estos descubrimientos.

\subsection{Ciencia}

La ciencia es la practica ejercida por una sociedad o individuos en base a cuestionarse el comportamiento de su entorno y el cuestionarse porque su ambiente funciona o se comporta de determinada maneja, en busca de entenderlo y poder saber como controlarlo o predecir en pro de poder aplicarlo en acciones que le traigan un beneficio a la sociedad a partir de aplicaciones tecnologicas dadas por la adquicision de los conocimientos obtenidos por la ciencia.

\subsection{Tecnologia}

La tecnologia consiste en la creacion y uso de herramientas elaboradas por grupos sociales o individuos, con el objetivo de facilitar o mejorar la eficiencia con que se realiza un proceso buscando traer un bienestar en las sociedades y facilitar trabajos tediosos o complejos.\\
Consiste en aplicar los conocimientos adquiridos por la ciencia, en medios que nos ayuden a solucionar necesidades complejas.\\
Es vital para el avance de la civilizacion ya que permite librarnos de procedimientos rutinarios y poder concentrarnos en la adquisicion de mas conocimiento y tecnologia.

\subsection{Sociedad}

La sociedad es un grupo de individuos que comparten entre si algunas costumbres o entornos que los hace pertenecer a un mismo grupo que colaboran entre ellos bajo el objetivo de crecer y mejorar sus vidas viviendo en comunidad.\\
Que vivan en un mismo entorno no implica que compartan las misma ideologias.

\section{Desarrollo}

\subsection{Problemas de la filosofia de la ciencia}

La filosofia de la ciencia consiste en el razonamiento del impacto de la ciencia en el ambiente de los individuos, ello implica que pueden haber multiples posiciones de multiples grupos e individuos en cuanto a las aplicaciones que se le den a los conocimientos adquiridos o descubiertos, ya que cada individuo se comporta de forma distinta y ello implica que razone y cuestione determinados temas de forma distinta a los demas, lo cual puede llevar a disputas sociales.
\\
Mientras una sociedad con sus respectivas costumbres y creencias pueden estar de acuerdo en cuanto las causas de una practica, por otra parte, otra socidad puede tener un pensamiento totalmente distinto sobre dichas practicas, lo cual puede traer polarizacion y disputas en el cuestionamiento de que camino tomar.
\\
\\
Un ejemplo actual son las telecomunicaciones. Parte de la sociedad esta de acuerdo en cuanto a los avances cientificos logrados en esta rama de la tecnologia ya que trae multiples beneficios en cuanto a acortar tiempos en las comunicaciones y facilitar muchas labores que podrian ser mas complejas sin estos avances tecnologicos. Por otra parte, algunos grupos insisten en que este tipo de tecnologia puede traer sintomas o enfermedades a los individuos de la sociedad. Todo esto lleva a que no se llegue a un mismo pensamiento colectivo y se polaricen los pensamientos sobre algunas practicas cientificas.
\\
Otro ejemplo son las pandemias. Diversos grupos son firmes al creer que un confinamiento o cuarentena es efectivo para evitar la propagacion de un virus o enfermedad, por otra parte, otros grupos sugieren que no causa gran efecto mas sin embargo si causa graves golpes en la economia de las sociedades.
\\
\\
Estos ejemplos demuestras las delimitaciones de la filosofia de la ciencia en cuanto a que camino escoger ante diversos temas de controversia.


\subsection{Delimitaciones del ambito de los estudios de la ciencia}

La ciencia a sido una rama fundamental en la historia de la humanidad, nos ha permitido lograr entender nuestro entorno y aplicar esos conocimientos para crear tecnologia que permita adquirir mas conocimientos o aplicarlos en construccion de nuevas tecnologias para mejorar el bienestar de las sociedad, ha permitido a sociedades mejorar en conocimientos, infraestructura y demas avances.

Hay multiples limitaciones en el estudio de las ciencias que muchas veces a frenado a las sociedades para lograr descubrir mas ciencia. 
\\
\\
Algunos limites pueden ser por falta de tecnologia, ya que algunas practicas cientificas pueden requerir de habilidades sobre humanas que no se han logrado llevar a cabo por atrazos tecnologicos, limites que puede que con el tiempo y con el descubrimiento de tecnologias mas avanzadas, lleguen a ser superados o tambien puede que sean limitaciones regidas por leyes que simplemente no hemos descubierto como alcanzarlas.
\\
\\
Otro tipo de limitacion es la economia. Para lograr descubrir algunos conocimientos cientificos, se requiere de maquinaria, tecnologia e infrastructura que demanda de recursos economicos altos para lograr su aplicacion, lo cual limita que se implemente un medio para avanzar en la ciencia.\\

\subsection{Delimitaciones del ambito de los estudios de la tecnologia}

La tecnologia, al igual que la ciencia, tiene diversas limitaciones en cuanto a su avance, de por si una de esas limitaciones es la misma ciencia.\\
La tecnologia muchas veces se puede ver frenada por la ciencia ya que esta puede requerir de conocimientos aun no alcanzadas para poder aplicarla. Un ejemplo aun algo futurista para nuestra epoca son los viajes intergalacticos, aun no contamos con los conocimientos astrofisicos para poder implementar una tecnologia que permita el transporte de sociedades a lugares lejanos de nuestro planeta.
\\
\\
Igualmente la economia es otro factor que frena mucho el avance en esta rama. Un ejemplo es la navegacion espacial, ya que los recursos necesarios para estos viajes suelen ser muy altos y actualmente no generan grandes contribuciones a la economia de la sociedad, por lo cual no se logra avanzar en la tecnologia de estas ramas.

\subsection{Delimitaciones del ambito de los estudios de la sociedad}

Los estudios de la sociedad consisten en el analisis del comportamiento de las comunidades y entender como interactuan y se relacionan entre individuos y sociedades.\\
\\
El estudio de la ciencia tiene las limitaciones de ser impredecible, dado que estan conformadas por individuos y el comportamiento y pensamiento de estos es impredecible, esto provoca que el analicis del comportamiento de los factores de una sociedad sea dificil de entender, factores como su economia, politica, educacion. ideologias y creencias.\\
\\
En pocas palabras, la principal limitacion del estudio de la sociedad es lo impredecible del comportamiento de sus individuos, ya que como se menciono anteriormente, el hecho de que compartan un entorno y cultura, no implica que razonen y crean en las mismas ideologias, lo cual provoca la diversidad y pensamientos diversos que muchas veces crean polariazaciones y llevan a las sociedades a tomas diversos caminos a seguir.

\subsection{Relacion de los ambitos del estudio con los problemas de la Filosofia de la ciencia}
    
Los problemas de la filosofia de la ciencia se relacionan bastante con los demas ambitos de estudio mencionados anteriormente, ya que los problemas de la filosofia de la ciencia dependen tanto de la ciencia, de la tecnologia y de la sociedad.\\

Los principales problemas de la filosofia se basan en el cuestionamiento del porque realizar ciencia y para que practicarla, de que sirve para nuestro avance tecnologico y que clase de beneficios o consecuencias puede traerle a nuestra forma de vivir en nuestras sociedades.
Por ello es necesario el estudio de la ciencia, tecnologia y sociedad, para poder entender las consecuencias que pueden traernos nuestras decisiones y escoger cual es mas conveniente en cuanto a beneficios.\\
Todos estos ambitos de estudio estan relacionados entre si, la ciencia se basa en entender el comportamiento de nuestro entorno, la tecnologia es la aplicacion de esos conocimientos para lograr beneficios de mayor confor y bienestar en nuestra sociedad. Todas estan relacionadas entre si pero no siempre pueden ser beneficiosas para la sociedad, muchas veces estos conocimientos o tecnologias mal aplicadas pueden llevarnos aaplicaciones que generen consecuencias poco beneficiosas o dañinas para la sociedad. Es hay donde entra en juego la filosofia de la ciencia, para decidir como seres racionales, que camino nos conviene mas para avanzar como una comunidad total sin traerle daños a nuestro entorno y nuestras forma de vivir, pero como fue mencionado anteriormente, como individuos podemos tener diversas formas de pensar y la polarizacion en nuestras sociedades en cuanto a pensamientos e ideologias nos llevan a disputas y no lograr acuerdos en comun.

\section{Concluciones}

Se concluye finalmente que el estudio de la ciencia, tecnologia y sociedad deben estar junto con el pensamiento de la filosofia de la ciencia que nos guie en cuanto a que caminos tomar que nos traigan un beneficio en comun sin traer consecuencias negativas. Son estudios que deben ir guiados en comun por la filosofia y el pensamiento de etica para no traer consecuencias negativas a nuestro entorno y al medio ambiente del que dependemos.\\

Se da a entender que hay diversas limitaciones en todos los campos de estudios y tambien en la filosofia, que aunque varias son limitaciones fisicas, tambien varias son limitaciones de nuestro sistema social o de nuestras ideologias, limitaciones que nos frenan muchas veces por no poder llegar a pensamientos comunes en cuanto a una idea o movimiento colectivo.