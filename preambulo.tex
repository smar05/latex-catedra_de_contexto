\usepackage[utf8]{inputenc}

\usepackage{graphicx}
\usepackage{subcaption}

%\usepackage{subfig}         % Multiples imagenes
\usepackage{vmargin}		% Paquete para margenes y formato de página
\setpapersize{A4}
\setmargins	{2.5 cm}     	% margen izquierdo
{0.5 cm}                 	% margen superior
{16.5cm}               		% anchura del texto
{24cm}             		    % altura del texto
{20pt}                		% altura de los encabezados
{1.2cm}               		% espacio entre el texto y los encabezados
{0pt}                		% altura del pie de página
{2cm}                 		% espacio entre el texto y el pie de página

% Preambulo
\usepackage{tikz, tkz-euclide} % hacer gráficos
\usepackage[american, european]{circuitikz} % hacer circuitos
\usepackage{siunitx} % Unidades SI \SI{}{} and \si{} commands for typesetting SI units

\usepackage[spanish, es-tabla, es-noshorthands]{babel}
\usepackage{amsmath}
\usepackage{multirow}
\usepackage{multicol}
\usepackage{tikz, tkz-euclide}
\usetikzlibrary{shapes.geometric, shapes.symbols, arrows, shadows}
\usepackage{graphicx}
\usepackage{float}
\usepackage{booktabs}
\usepackage{circuitikz}
\usepackage{adjustbox}
\usepackage[many]{tcolorbox}
    \tcbuselibrary{theorems}

\usepackage{hyperref}
\usepackage[acronym]{glossaries}
\makeglossaries
%\include{glosario}
\include{acronimos}
\usepackage{biblatex}
\addbibresource{ref.bib}
%\tikzset{>=latex}

\usepackage{pgfplots}
\usetikzlibrary{pgfplots.smithchart}
\usepgfplotslibrary{colorbrewer}
\pgfplotsset{cycle list/Set1-4}
\usepgfplotslibrary{polar}
\pgfplotsset{compat=newest}
\usepgfplotslibrary{smithchart}
\usepackage{siunitx} 
\usepackage{tikz, tkz-base, tkz-fct, pgfplots}
\usepackage{colortbl}
\usepackage{steinmetz}
\usetikzlibrary{bending}
\pgfplotsset{compat=1.13}
\definecolor{micolor}{rgb}{0.9,0.5,0.9}
\ctikzset{current  arrow color/.initial=red}

\usepackage{xparse}
\DeclareDocumentCommand{\newdualentry}{ O{} O{} m m m m } {
  \newglossaryentry{gls-#3}{name={#5},text={#5\glsadd{#3}},
    description={#6},#1
  }
  \makeglossaries
  \newacronym[see={[Glossary:]{gls-#3}},#2]{#3}{#4}{#5\glsadd{gls-#3}}
}


\usepackage{amsmath, amsthm, amsfonts}
\def\RR{\mathbb{R}}
\def\ZZ{\mathbb{Z}}

% De la misma forma se pueden definir comandos con argumentos. Por
% ejemplo, aquí definimos un comando para escribir el valor absoluto
% de algo más fácilmente.
%--------------------------------------------------------------------------

% declaracion de unidades no SI
\DeclareSIUnit{\pf}{pF}
\DeclareSIUnit{\pH}{pH}
\DeclareSIUnit{\nh}{nH}
\DeclareSIUnit{\nf}{nF}
\DeclareSIUnit{\uf}{{\micro}F} 
\DeclareSIUnit{\uh}{{\micro}H} 
\DeclareSIUnit{\ms}{mS}
\DeclareSIUnit{\second}{s}
\DeclareSIUnit{\MHz}{MHz}
\DeclareSIUnit{\dB}{dB}
\graphicspath{ {imagenes/} }

%-----------------------------------------------------------------------------
%               Entorno Para ejemplos 
%-----------------------------------------------------------------------------

 
 %                  Entorno para ecuaciones de ejemplos
 %--------------------------------------------------------------------------
\usepackage{environ}               % Define entornos (ecuaciones de ejemplos)
\NewEnviron{meq}{%
    \begin{equation}
    \scalebox{1}{$\BODY$}           % Se utiliza el paquete "graphix"
    \end{equation}
    }
%------------------------------------------------------------------------------
%                   Entorno de para los diagramas de flujos, como  "Mason"   
%------------------------------------------------------------------------------
\usetikzlibrary{decorations.markings,arrows.meta}
\tikzset
  {midarrow/.style={decoration={markings,mark=at position 0.5 with
     {\arrow[xshift=2pt, purple]{Latex[length=6pt,#1]}}},postaction={decorate}}
  }
  
  \newcommand{\txb}[1]{\small\sffamily #1}
\def\RR{\mathbb{R}}
\def\ZZ{\mathbb{Z}}

%------------------------------------------------------------------------
%           Macros para las redes red1 para grandes (pocos componentes 0-12)
%           pequeños, red2 medianas (varios componentes 0-16)
%           y red3 grandes pequeñas (muchos componentes 0-20)
%-------------------------------------------------------------------------
\newenvironment{red0}[1][0.8]
    {
   \begin{adjustbox}{width=0.4\textwidth}
    \begin{circuitikz}[scale=#1]
    }
    { 
    \end{circuitikz}
   \end{adjustbox} 
    }

\newenvironment{red1}[1][0.7]
    {  \begin{subfigure}[b]{ 0.45\textwidth}
   \begin{adjustbox}{width=0.6\textwidth}
    \begin{circuitikz}[scale=#1]
    }
    { 
    \end{circuitikz}
   \end{adjustbox} 
     \caption{}
    \end{subfigure} 
    }
    
\newenvironment{red2}[1][0.7]{
        \centering
    \begin{subfigure}[b]{0.45\textwidth}
    \begin{adjustbox}{width=0.8\textwidth}
    \begin{circuitikz}[scale=#1]
    }
    { 
    \end{circuitikz}
    \end{adjustbox} 
    \caption{}
    \end{subfigure} 
    }
\newenvironment{red3}[1][0.7]
    {\begin{subfigure}[b]{0.45\textwidth}
    \begin{adjustbox}{width=1\textwidth}
    \begin{circuitikz}[scale=#1]
    }
    { 
    \end{circuitikz}
    \end{adjustbox} 
     \caption{}
    \end{subfigure} 
    }

%------------------------------------------------------------------------
%           Macros para las redes re1 pequeños, rd2 medianos y red3 grandres
%-------------------------------------------------------------------------
\newenvironment{red1a}[1][0.7]{
        
        \begin{adjustbox}{scale=0.8}
        \begin{circuitikz}[scale=#1]
    }
    { 
        \end{circuitikz}
        \end{adjustbox} 
    %\end{center}
    }
    
    \newenvironment{reda}[1][1]
    {\begin{subfigure}[b]{0.4\textwidth}
   \begin{adjustbox}{width=1\textwidth}
    \begin{circuitikz}[x=#1cm,y=#1cm]
    \tikzset{font={\fontsize{15pt}{12pt}\selectfont}}
    %\ctikzset{bipoles/length=25mm}
    \centering
    }
    { 
    \end{circuitikz}
    \end{adjustbox}
    \end{subfigure} 
    }
    
%\ctikzset{bipoles/ammeter/text rotate/.initial=0, rotation/.style={bipoles/ammeter/text rotate=#1},
%}
\newenvironment{redb}[1][0.8]
    {\begin{subfigure}[b]{0.45\textwidth}
   \begin{adjustbox}{width=1\textwidth}
    \begin{circuitikz}[x=0.8cm,y=#1cm]
    \tikzset{font={\fontsize{15pt}{15pt}\selectfont}}
    \ctikzset{bipoles/length=20mm}
    }
    { 
    \end{circuitikz}
    \end{adjustbox} 
     \caption{}
    \end{subfigure} 
    }
%------------------------------------------------------------------------
%          Entorno para los ejemplos
%-------------------------------------------------------------------------
%\usepackage{draftwatermark}
%\SetWatermarkText{\textsc{jpoveda@}} %
% \newtcbtheorem[auto counter, number within = section]{ejemplo}{Ejemplo}%
% 	{   enhanced,
% 	arc=2mm,
% 	interior style={},
% 	attach boxed title to top center= {yshift=-\tcboxedtitleheight/4},
% 	borderline={0.5mm}{0mm}{gray!20!white,dashed},
% 	fonttitle=\bfseries,
% 	fontupper=\itshape,
% 	colbacktitle=white,coltitle=black,
% 	boxed title style={size=normal,colframe=purple!50,boxrule=1pt}
% 	}
% {th}

\newtcbtheorem[auto counter, number within = section]{ejemplo}{Ejemplo CA}%
	{   enhanced,
	arc=2mm,
	interior style={},
	attach boxed title to top center= {yshift=-\tcboxedtitleheight/4},
	borderline={0.5mm}{0mm}{gray!20!white,dashed},
	fonttitle=\bfseries,
	fontupper=\itshape,
	colbacktitle=white,coltitle=black,
	boxed title style={size=normal,colframe=purple!50,boxrule=1pt}
	}
{th}
